\section{Introduction}
\label{sec:intro}

Over the past few years, we have been the beneficiaries of
two quietly game-changing technological revolutions: the coming
of age of low-power processors and computer vision.  While
much of the world's attention has been devoted to Instagram
and Snapchat, cell phone processors silently crossed the 1Ghz
and dual-core marks; recently, Samsung announced an eight core
processor intended for use in high-end smartphones~\cite{exynos}.
Where, as recently as 2007, the flagship cell phone coming to market
was a flip-phone~\cite{razr}, today's smartphones now rival tablets - 
or even laptops - in their computational capabilities.  For use in such
platforms, these processors are also cheap and exceedingly power-efficient.

Although they have mainly seen use in smartphones, these CPUs have
found their way into a variety of other environments where cost and
energy-efficiency are requirements.  One device, the Raspberry Pi,
leverages these developments in inexpensive hardware to supply a
general-purpose computer at a price well under \$50 and in a package
little larger than a pack of playing cards~\cite{pi:faqs}. Online guides exist
for turning the device into everything from a media player appliance
or a sophisticated sensor to a fully-functional desktop machine~\cite{pi:ideas}.
We envision a new application for these inexpensive yet powerful computers
based on recent developments in computer vision.

Computer vision, the process of algorithmically interpreting images to
extract knowledge about their content, has, in the past decade, grown
from a pure research endeavor into a useful toolkit for developers.
OpenCV is an open-source computer vision library compatible with all major
desktop operating systems and with implementations for C++, Python,
Java, and other languages.  It provides a convenient interface for
programmers with limited domain-specific expertise to develop applications
with significant vision components~\cite{cv:about}. Beyond simple ease-of-use,
the library's vision algorithms are highly optimized to be performance-friendly.
OpenCV's success is a major milestone for computer vision research, marking
its ascendance into the ranks of core computer science knowledge.

Connecting the dots, computer vision is getting ever less demanding as
cheap processors are becoming increasingly capable.  We see a world
of possibilities in the intersection of these two trend lines, pairing cameras
with Raspberry Pi-class devices to unlock the imaginations of developers
and enable a variety of new applications.  In this paper, we describe the
design and implementation of a framework for distributed systems of these
computer-equipped cameras that:

\begin{enumerate}
    \setlength{\itemsep}{5pt}
    \item Uses inexpensive, commodity computer and network hardware.
    \item Allows developers to write simple vision algorithms that are applied across the system.
    \item Executes these algorithms efficiently while automatically managing the limited resources of these inexpensive computers.
    \item Makes the results of these algorithms accessible in a convenient, high-level manner.
\end{enumerate}

We first describe the original inspiration for this framework, smart security cameras
(Section 2), and an initial eager, "push" architecture that supports this application (Section 3).
We then explain an extension of this original framework into a lazy, "pull" architecture with an
HTTP-based interface and better resource utilization (Section 4).