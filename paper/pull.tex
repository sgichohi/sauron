\section{Pull Architecture}
\label{sec:pull}

\paragraph{Limitations of the Push Architecture.}
The most significant weakness of the push architecture is
that it eagerly applies a transformation to all frames, creating
many sendables that, due to network limitations, may never
actually be transmitted.  This unnecessary computational work is
potentially significant on a device like a Raspberry Pi. The
push architecture also presents a difficult-to-use
interface to a programmer writing a recipient server, which
must efficiently handle a continuous flow of sendables.
These two deficiencies inspired an extension of the original
architecture into one which lazily applies transformations and
presents an HTTP-based interface to recipients, resulting in a
more efficient and usable framework.

\paragraph{Design.}
Although performing all transformations lazily would achieve
optimal efficiency by only doing work that is absolutely necessary,
solely permitting lazy transformations would render global sendable
reordering impossible.  We therefore took a hybrid approach
where a user supplies two sets of transformations.  The first,
eager transformation is applied as in the push architecture.
Rather than doing all of the work, the eager transformation
should be written so as to efficiently do any reordering and
data size reduction that must be applied globally.  These
\emph{intermediate sendables} are inserted into a database, where
they are accessible to a number of more work-intensive, lazy
transformations.  Data recipients can then request the final
results of these transformations over HTTP.  This compromise
between laziness and eagerness maintains the best aspects
of the original model while compensating for its weaknesses.

\paragraph{Programmer Interface}
These changes represent a pure extension to the original framework.
Rather than pushing sendables over the network, the original model
was modified to insert them, along with other metadata, into a SQLite
database.  This database is made accessible to a webserver interface
written in Python.  Users of the framework write Python functions
that encapsulate different queries a recipient might make on a lazy
transformation, such as performing it on the highest-scored sendable
available or sendables created between 2:00pm and 3:00pm.  This
Python section of the framework exports URLS of the format:

\texttt{\\http://[ip]/do/[function]?[arguments]\\}

The \texttt{function} component of the URL refers to the name of
the Python function to be executed.  Recipients supply arguments to the method using
the URL's query string.  Users of the framework are free to either
pass the results of the transformation in response to the original HTTP
message or to respond with a second URL from which the results are accessible.
Since Python has first-class functions, programmers can easily 
compose and combine transformations.

\paragraph{Enhancements}

In addition to optimizing the performance of the framework,
this extension supports several additional enhancements.  Users may 
define multiple transformations that run in parallel on each camera and
examine cameras from any number of recipients.  These transformations
have free reign over the database of sendables, permitting
the framework to transparently support sendable-specific queries that
would previously have been impossible.   The pull architecture provides a
convenient interface at a high level of abstraction for the framework, making
it far easier to use than the push architecture alone.

Finally, the push architecture presents a much more useful interface
for a user wishing to examine the state of a camera over the interent.
To do so with the push architecture would have required a viewer to receive
and store all frames that a camera supplied, which would be highly inconvenient on
a smartphone app.  The pull architecture permits a user to perform queries on
the state of a camera without receiving an overwhelming amount of data or
modifying the state of the camera in any way.
