\section{Related Work}
\label{sec:related}

There has been significant previous work on video analysis and keyframe
detection in the computer vision community. For example, Girgensohn and
Boreczky \cite{related1} accomplish this process by applying a clustering
algorithm to the set of frames from a video and selecting a representative
frame from each cluster. Wolf \cite{related2} first divides the video into shots by detecting 
camera cuts and transitions, and then looks for local minimums of object motion
within each shot, presenting these as the keyframes. These approaches are
not a good fit for our problem, as they either only work offline, or are based
on assumptions about how videos are shot that only hold true for media
professionally produced for entertainment. 

There has also been research into applying computer vision to analyze
security footage, but it focuses on semantic analysis of large archives of
video, rather than on determining individual frames that are not important.
For example, Regazzoni’s work \cite{related3} finds abandoned objects in
video footage to help police searching events like, for example, the moment
when a bomb was planted. 

For the Feature Delta transformer, we used an approach based on applying
the SIFT feature detector \cite{related4} to find features within each frame
and then assigning importance to frames where new features arise
or existing features move significantly. This approach is simple and fast,
can use existing implementations of the SIFT algorithm, and is
not based on a particular semantic interpretation of the scene.


