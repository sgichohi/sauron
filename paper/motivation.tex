\section{Motivation}
If we were to consider the essential requirements of an
effective security camera, what would our list look like?
We know that they would have to take pictures and
rapidly transmit them to a central server.  We would desire
this process to be resilient to network and camera failure,
storing frames persistently and transmitting them as soon
as the network comes back up.  In the event that the network
is congested, they should send the most "important" frames
first, ensuring that critical information reaches the server as
soon as possible.  Physically, they should be easy to install and remove,
providing flexibility and mobility.  Above all, they should be cheap.

It may seem surprising to note that commercially available
security camera solutions satisfy only one of these requirements:
they take pictures.  These systems rely on dedicated, proprietary
physical links or wireless networks, limiting flexibility and increasing
costs.  They demand favorable environments in which congestion
is non-existent, connection speeds are always sufficient, and all processing
can be done instantaneously at the server.  To guarantee these
conditions, manufacturers limit the number of "channels," and therefore
simultaneous cameras, their systems will support.  Even with all of these 
limitations, security camera systems are still quite expensive, with a
\$100/camera price at the lower end of the market.  Most camera
packages tout features like camera quality and access over the internet and
via smartphone apps.

All told, this perfect confluence of ideal circumstances does not hold in
many situations.  Conditions where cameras must be quick to install or even mobile, the 
connection medium is shared, and the link itself is unreliable abound in a variety 
of contemporary settings where video monitoring is essential.  From outdoor 
events to war zones, many environments contain one or more of these adverse 
components that traditional security systems could not overcome.  Furthermore, 
an ideal environment quickly becomes adverse when its owner loses physical 
control, as evidenced in the recent attack at a shopping mall in Nairobi among 
numerous other incidents in the past few years.

We believe that, using an array of Raspberry Pi devices with camera attachments
(\$70 on Amazon at the time of writing), we can develop - at lower cost - a more
robust, flexible, and intelligent security system that is able to continue to meet
these more stringent requirements.  A camera would run feature-detection on each frame it captures,
comparing it to its predecessor and scoring it based on the difference.  Frames with higher
scores would be sent first, with all others cached on persistent memory
until there is enough network bandwidth to send them.  The system could communicate
over a standard 802.11 wireless network, rendering cameras mobile within wireless range
while imposing no strict limit on the number of channels available.  Our framework was
designed with this example in mind, although it has been generalized to support arbitrary
distributed computer vision.

\label{sec:motivation}